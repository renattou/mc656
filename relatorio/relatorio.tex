\documentclass[a4paper,11pt,twoside]{article}
\usepackage[portuguese]{babel}
\usepackage[utf8]{inputenc}
\usepackage{mathtools}
\usepackage{amssymb}

\title{Relatorio MC658}
\author{Tiago L. Gimenes - 118827\\
        Renato L. Vargas - 118557}

\begin{document}
\maketitle

\section{Introdução}
TODO

\section{Metaheurística}
TODO

\section{Modelo PLI}
\subsection{Entrada e constantes}
Seja $n$ a quantidade de cenas, $m$ a quantiade de atores e $b$ a quantidade de
dias. Como somente uma cena é gravada por dia, temos que $n = b$. Sejam $N=\{1..n\}$,
$M = \{1..m\}$ e $B=\{1..b\}$ os conjuntos que representam as cenas, atores e dias
respectivamente. Da entrada do problema, podemos definir as seguintes constantes:

\begin{equation}
  t_{ij} = \begin{dcases*}
              1 & se ator $i \in M$ esta na cena $j \in N$ \\
              0 & se nao.
           \end{dcases*}
\end{equation}

\begin{equation}
  c_{i} = \begin{dcases*} k \in \mathbb{N} & salario diario do ator $i \in M$ \end{dcases*}
\end{equation}

\begin{equation}
  s_{i} = \begin{dcases*} \sum_{j \in N} t_{ij} \in \mathbb{N} & numero de cenas que o ator $i \in M$ participa\end{dcases*}
\end{equation}

\subsection{Variáveis}
Para formular esse problema, utilizaremos as seguintes variaveis:

\begin{equation}
  p_{jl} = \begin{dcases*}
              1 & se cena $j \in N$ esta no dia $l \in B$ \\
              0 & se nao.
           \end{dcases*}
  \label{pjl}
\end{equation}

\begin{equation}
  x_{il} = \begin{dcases*}
              1 & se ator $i \in M$ esta no dia $l \in B$ \\
              0 & se nao.
           \end{dcases*}
  \label{xil}
\end{equation}

\begin{equation}
  e_{i} = \begin{dcases*} k \in B & dia em que ator $i \in M$ começa a trabalhar \end{dcases*}
  \label{ei}
\end{equation}

\begin{equation}
  d_{i} = \begin{dcases*} k \in B & dia em que ator $i \in M$ termina de trabalhar \end{dcases*}
  \label{di}
\end{equation}

\begin{equation}
  h_{i} = \begin{dcases*} k \in B & dias de espera do ator $i \in M$ \end{dcases*}
  \label{hi}
\end{equation}

Intuitivamente, as variaveis (\ref{pjl}) representam se uma cena esta em certo dia
ou nao, ou seja, ela representa a ordem das cenas. Visto que há uma variavel para
cada par de indices $jl$, temos $O(n*b) = O(n^2)$ variáveis do tipo $p_{jl}$. As
variáveis do tipo (\ref{xil}) representam se um determinado ator fará uma cena
em um determinado dia. Neste caso, há $O(m*b) = O(m*n)$ variáveis do tipo $x_{il}$.

As variáveis do tipo (\ref{ei}) e (\ref{di}) representam, respectivamente, os dias em que o ator
começa e termina de participar das cenas, ou seja, elas marcam a participação
mais a esquerda e mais a direita de um ator em uma ordenação das cenas. Como há
uma variável de cada tipo para cada ator, há $O(m)$ variáveis deste tipo no programa.

As variáveis (\ref{hi}) representam os dias de espera de um determinado ator, como
descrito no enunciado. Há também $O(m)$ variáveis deste tipo no programa linear.
\subsection{Restrições}
Há basicamente sete conjuntos de restrições abaixo. Cada restrição pode ser encontrada
à esquerda do símbolo de chaves. À direita é mostrado o dominio das variaveis livres
e, logo após, é dada uma breve descrição da restrição:
\begin{equation}
  \sum_{j \in N} p_{jl} = 1 \begin{dcases*} \forall l \in B & Uma cena por dia \end{dcases*}
  \label{cena_dia}
\end{equation}

\begin{equation}
  \sum_{l \in B} p_{jl} = 1 \begin{dcases*} \forall j \in N & Um dia por cena \end{dcases*}
  \label{dia_cena}
\end{equation}

\begin{equation}
  \sum_{j \in N} t_{ij} p_{jl} = x_{il} \begin{dcases*} \forall i \in M, l \in B & ator esta no dia se estiver na cena desse dia \end{dcases*}
  \label{actor_cene_day}
\end{equation}

\begin{equation}
  e_{i} \leq n + (l-n)x_{il} \begin{dcases*} \forall i \in M, l \in B & primeiro dia do ator \end{dcases*}
  \label{first_day}
\end{equation}

\begin{equation}
  d_{i} \geq lx_{il} \begin{dcases*} \forall i \in M, l \in B & ultimo dia do ator \end{dcases*}
  \label{last_day}
\end{equation}

\begin{equation}
  d_{i} - e_i + 1 - s_i = h_i \begin{dcases*} \forall i \in M & dias ociosos \end{dcases*}
  \label{wait}
\end{equation}

\begin{equation}
  \sum_{i=1}^j p_{in} \leq 1 - p_{j1} \begin{dcases*} \forall j \in N & quebra de simetria \end{dcases*}
  \label{sem_simetria}
\end{equation}

\section{Branch and Bound}
TODO

\section{Experimentos}
TODO

\section{Conclusão}
TODO

\end{document}
