\documentclass[a4paper,11pt]{article}
  \usepackage[a4paper,left=3cm,right=3cm,top=3cm,bottom=3cm]{geometry}
\usepackage[portuguese]{babel}
\usepackage[utf8]{inputenc}
\usepackage{mathtools}
\DeclarePairedDelimiter{\ceil}{\lceil}{\rceil}
\DeclarePairedDelimiter{\floor}{\lfloor}{\rfloor}
\usepackage{amssymb}
\usepackage{longtable}

\title{Relatório MC658}
\author{Tiago L. Gimenes - 118827\\
        Renato L. Vargas - 118557}

\begin{document}
\maketitle

\section*{Introdução}
Neste trabalho tentamos atacar o problema de ordenação de cenas através de três
abordagens diferentes. Primeiramente com algoritmos exatos (PLI e B\&B) e em seguida
uma metaheurística (algoritmo genético). O algoritmo genético se mostrou satisfatório,
mas em vários casos ficou preso em locais mínimos em pouco tempo de execução. O algoritmo de B\&B
proposto se mostrou capaz de resolver instâncias relativamente grandes (com 18 cenas),
enquanto que o PLI se mostrou como um bom validador de resultados para os outros
algoritmos.

\section{Metaheurística}
A metaheurística implementada foi o algoritmo genético já estudado em aula.
As especificidades da nossa implementação são descritas nessa seção.

\subsection{Fitness}
A codificação de um indivíduo é dada sempre por uma solução completa.
A função de fitness de um indivíduo é então exatamente o custo desta solução.

\subsection{Manutenção}
Inicialmente uma população é inicializada com tamanho $N_{pop}$, onde $N_{pop}-1$ indivíduos são
inicializados com soluções randômicas e $1$ indivíduo é inicializado com um algoritmo
guloso, já gerando possivelmente um indivíduo mais apto.

A cada geração o indivíduo mais apto é sempre preservado. São escolhidos então $(N_{pop}-1)/2$
pares pelo método da roleta visto em sala. Esses pares são cruzados, gerando dois novos
indivíduos. Esses novos indivíduos sofrem então possíveis mutações. Dessa forma, ao final
de uma etapa de evolução, temos novamente $N_{pop}$ indivíduos.

\subsection{Cruzamento}
O cruzamento é sempre feito entre dois indivíduos, gerando dois outros novos.
Existem três variações de cruzamento e cada uma delas possui $1/3$ de chance de ocorrer.
As três variações são baseadas em métodos principal que é inspirado pelo método PBX
para o problema do Caixeiro Viajante.

Primeiramente é sorteado um número $C$ limitado por duas constantes
$C_{min}$ e $C_{max}$, onde $C_{min}, C_{max} = \{0..1\}$ e $C_{min} \leq C_{max}$.
Sendo $n$ o número de cenas, um trecho contínuo de tamanho $C_n = \ceil{C * n}$ é cruzado.

Dado dois indivíduos $I_1$ e $I_2$, são gerados $I_3$ e $I_4$. Fora do trecho escolhido,
as cenas de $I_3$ são iguais à $I_1$ e as de $I_4$ à $I_2$.
Dentro do trecho, as cenas de $I_3$ são preenchidas com as cenas na ordem que estão em $I_2$,
ignorando cenas que já estão em $I_3$ fora do trecho. O mesmo é feito para $I_4$,
usando as cenas na ordem que estão em $I_1$.

As variações consistem em começar o trecho no começo, começar em uma posição aleatória ou
terminar no final da solução.

\subsection{Mutação}
A mutação é sempre feita em cima de um indivíduo, fazendo uma quantidade limitada
de swaps entre cenas da solução, garantindo que as soluções geradas são sempra válidas.
Cada swap é feito com uma probabilidade $R_{mut} = \{0..1\}$.
Existem três variações de mutação e cada uma delas possui $1/3$ de chance de ocorrer.

A primeira variação consiste em percorrer cena por cena até a penúltima, fazendo um possível
swap entre esta cena e uma outra sorteada que deve estar após ela. São feitos no máximo $n-1$ swaps.

A segunda variação consiste em percorrer cena por cena até a metade, fazendo um possível swap
entre esta cena e a outra na mesma posição na solução inversa. São feitos no máximo $\floor{n/2}$ swaps.

A terceira variação consiste em percorrer cena por cena até a penúltima, fazendo um possível swap
entre esta cena e a cena seguinte. São feitos no máximo $n-1$ swaps.

\section{Modelo PLI}
\subsection{Entrada e constantes}
Seja $n$ a quantidade de cenas, $m$ a quantidade de atores e $b$ a quantidade de
dias. Como somente uma cena é gravada por dia, temos que $n = b$. Sejam $N=\{1..n\}$,
$M = \{1..m\}$ e $B=\{1..b\}$ os conjuntos que representam as cenas, atores e dias
respectivamente. Da entrada do problema, podemos definir as seguintes constantes:

\begin{equation}
  t_{ij} = \begin{dcases*}
              1 & se ator $i \in M$ está na cena $j \in N$ \\
              0 & se não.
           \end{dcases*}
  \label{tij}
\end{equation}

\begin{equation}
  c_{i} = \begin{dcases*} k \in \mathbb{N} & salário diário do ator $i \in M$ \end{dcases*}
\end{equation}

\begin{equation}
  s_{i} = \begin{dcases*} \sum_{j \in N} t_{ij} \in \mathbb{N} & número de cenas que o ator $i \in M$ participa\end{dcases*}
\end{equation}

\subsection{Variáveis}
Para formular esse problema, utilizaremos as seguintes variáveis:

\begin{equation}
  p_{jl} = \begin{dcases*}
              1 & se cena $j \in N$ está no dia $l \in B$ \\
              0 & se não.
           \end{dcases*}
  \label{pjl}
\end{equation}

\begin{equation}
  x_{il} = \begin{dcases*}
              1 & se ator $i \in M$ está no dia $l \in B$ \\
              0 & se não.
           \end{dcases*}
  \label{xil}
\end{equation}

\begin{equation}
  e_{i} = \begin{dcases*} k \in B & dia em que ator $i \in M$ começa a trabalhar \end{dcases*}
  \label{ei}
\end{equation}

\begin{equation}
  d_{i} = \begin{dcases*} k \in B & dia em que ator $i \in M$ termina de trabalhar \end{dcases*}
  \label{di}
\end{equation}

\begin{equation}
  h_{i} = \begin{dcases*} k \in B & dias de espera do ator $i \in M$ \end{dcases*}
  \label{hi}
\end{equation}

Intuitivamente, as variáveis (\ref{pjl}) representam se uma cena está em certo dia
ou não, ou seja, ela representa a ordem das cenas. Visto que há uma variável para
cada par de índices $jl$, temos $O(n*b) = O(n^2)$ variáveis do tipo $p_{jl}$. As
variáveis do tipo (\ref{xil}) representam se um determinado ator fará uma cena
em um determinado dia. Neste caso, há $O(m*b) = O(m*n)$ variáveis do tipo $x_{il}$.

As variáveis do tipo (\ref{ei}) e (\ref{di}) representam, respectivamente, os dias em que o ator
começa e termina de participar das cenas, ou seja, elas marcam a participação
mais a esquerda e mais a direita de um ator em uma ordenação das cenas. Como há
uma variável de cada tipo para cada ator, há $O(m)$ variáveis deste tipo no programa.

As variáveis (\ref{hi}) representam os dias de espera de um determinado ator, como
descrito no enunciado. Há também $O(m)$ variáveis deste tipo no programa linear.
\subsection{Restrições}
Há basicamente sete conjuntos de restrições abaixo. Cada restrição pode ser encontrada
à esquerda do símbolo de chaves. À direita é mostrado o domínio das variáveis livres
e, logo após, é dada uma breve descrição da restrição:
\begin{equation}
  \sum_{j \in N} p_{jl} = 1 \begin{dcases*} \forall l \in B & Uma cena por dia \end{dcases*}
  \label{cena_dia}
\end{equation}

\begin{equation}
  \sum_{l \in B} p_{jl} = 1 \begin{dcases*} \forall j \in N & Um dia por cena \end{dcases*}
  \label{dia_cena}
\end{equation}

\begin{equation}
  \sum_{j \in N} t_{ij} p_{jl} = x_{il} \begin{dcases*} \forall i \in M, l \in B & ator está no dia se estiver na cena desse dia \end{dcases*}
  \label{actor_cene_day}
\end{equation}

\begin{equation}
  e_{i} \leq n + (l-n)x_{il} \begin{dcases*} \forall i \in M, l \in B & primeiro dia do ator \end{dcases*}
  \label{first_day}
\end{equation}

\begin{equation}
  d_{i} \geq lx_{il} \begin{dcases*} \forall i \in M, l \in B & último dia do ator \end{dcases*}
  \label{last_day}
\end{equation}

\begin{equation}
  d_{i} - e_i + 1 - s_i = h_i \begin{dcases*} \forall i \in M & dias ociosos \end{dcases*}
  \label{wait}
\end{equation}

\begin{equation}
  \sum_{i=1}^j p_{in} \leq 1 - p_{j1} \begin{dcases*} \forall j \in N & quebra de simetria \end{dcases*}
  \label{sem_simetria}
\end{equation}

As restrições (\ref{cena_dia}) e (\ref{dia_cena}) fazem com que haja somente uma
cena por dia e um dia por cena respectivamente. Há um total de $O(n)$ restrições
de cada tipo

As restrições (\ref{actor_cene_day}) realizam a semântica das variáveis (\ref{xil}).
Note que, como há somente uma cena por dia e que as variáveis (\ref{pjl}) e (\ref{tij})
são booleanas, o resultado da soma do lado esquerdo também é booleano. Há $O(m*n)$
restrições desse tipo.

As restrições (\ref{first_day}) e (\ref{last_day}) contam, respectivamente, o
primeiro e último dia em que o ator gravará uma cena. Note que, como há $O(n)$
restrições desse tipo para cada ator, essas restrições são equivalentes a um
mínimo para o caso de (\ref{first_day}) e a um máximo para o caso de (\ref{last_day}).
Há um total de $O(m*n)$ restrições desse tipo.

Restrições do tipo (\ref{wait}) contabilizam os dias ociosos de cada ator, havendo
somente $O(m)$ restrições desse tipo.

Somente com as restrições de \ref{cena_dia} à \ref{wait}, uma solução do programa
linear, é solução do problema de ordenação de cenas. As retrições \ref{sem_simetria}
entram com o intuito de quebrar simetrias na solução e então, fazer com que o PLI
evite revisar algumas soluções, aumentando o tamanho das instâncias resolvidas pelo PLI.
Essas restrições evitam soluções espelhadas em torno da cena cental. Há $O(n)$ restrições
deste tipo.

\subsection{Função objetivo}
A função objetivo pode ser formulada facilmente em função das variáveis e constantes
seguindo o enunciado:

\begin{equation}
  \sum_{i \in M} c_i h_i
\end{equation}

\subsection{Formulação final}
A formulação final consta com todas as $O(nm + n^2)$ variáveis e as $O(nm)$
restrições descritas acima.

\section{Branch and Bound}
A implementação do algoritmo de \textit{branch and bound} seguiu os passos principais
descritos no enunciado, com as diferenças descritas nesta seção.

\subsection{Inicialização da solução primal}
Opostamente à dica do enunciado de se fazer uma busca em profundidade na árvore
de soluções, para se achar uma solução primal inicial, utilizamos a metaheurística
implementada neste trabalho (algoritmo genético). A metaheurística é executada
por 100ms antes da execução do algoritmo de \textit{branch and bound}.

Utilizamos a metaheurística com o intuito de que ela poderia proporcionar uma
boa solução primal com relativamente pouco esforço computacional. Essa premissa
se mostrou válida em alguns casos, onde a solução encontrada pela metaheurística
não estava tão distante da solução ótima, mas se mostrou falsa em outros casos
onde a metaheurística ficou presa em mínimos locais.

\subsection{Exploração da arvore de soluções}
A árvore de soluções foi explorada através de uma estratégia de \textit{best bound},
onde um \textit{heap} foi utilizado para a manutenção da fila de prioridade. A prioridade
desta fila é medida através do menor limitante inferior de um nó da árvore, com empates favorecendo
nós mais profundos.

Também com o intuito de se podar mais nós, quando um nó é explorado, para cada filho,
a solução parcial é completada através de um algoritmo guloso rápido para tentar se obter
limitantes primais melhores. Caso o custo da solução gulosa seja igual à do limitante
inferior para aquele nó, não é necessaria a exploração deste ramo. Caso a solução gulosa seja melhor
do que a melhor solução encontrada até o momento, a melhor solução é atualizada
para a gulosa.

\subsection{Heurística para obtenção do empacotamento $Q$}
Seguindo a dica do enunciado de utilizar um algoritmo guloso rápido com o intuito
de maximizar o tamanho do empacotamento $Q$, propomos um algoritmo que tenta
adicionar à $Q$ cenas com o menor número de atores possível, minimizando a probabilidade
de conflito de atores com outras cenas de $Q$, maximizando o empacotamento.

Sem perda de generalidade, supomos que estamos calculando o
limitante $k_3$, logo o conjunto de atores válidos a ser procurado o empacotamento
de cenas $Q$ é $B_E(P)$. Seja $C$ o conjunto de cenas que ainda não foram programadas nem à esquerda nem
à direita na solução parcial.

Primeiramente começamos a calcular o custo de cada cena $c \in C$, de acordo com
o enunciado, somando os custos dos atores de $B_E(P)$ presentes na cena $c$. Também
é calculado nessa parte quantidade de atores de $B_E(P)$ na cena $c$.

Com os custos e quantidade de atores em cada cena calculados, começamos por ordená-las
em ordem não decrescente de frequência de aparição dos atores, eliminando
as cenas com custo igual a zero. Percorremos então o vetor de cenas ordernado,
adicionando cenas à $Q$ sempre que possível, ou seja, tal que os atores presentes
na cena atual não estejam ainda presentes em nenhuma outra cena de $Q$, respeitando as condições
do enunciado.

Por fim, o empacotamento $Q$ é ordenado em ordem não crescente de custo, e retornado.

Mudando o conjunto $B_E(P)$ por $B_D(P)$ podemos calcular o empacotamento para o
limitante $k_4$ utilizando o mesmo algoritmo descrito acima.

\section{Experimentos}
Os experimentos a seguir foram executados em um Intel Core i7-4700MQ 2.40GHz com
16GB de memória RAM e compilados com GCC 7.2.0.

\subsection{Metaheurística}
Para os experimentos do algoritmo genético foram utilizados os valores
$N_{pop} = 25$, $C_{min} = 0.5$, $C_{max} = 0.8$ e $R_{mut} = 0.01$
para as variáveis anteriormente definidas. Os resultados podem ser vistos na Tabela \ref{meta}.
Como não sabemos se a metaheurística achou uma solução ótima, rodamos sempre até o tempo limite,
como pode ser observado na tabela.
Durante a execução do programa também são mostrados resultados geração a geração para melhor acompanhamento.

\begin{longtable}{c c c}
  Dataset & Função Objetivo & Tempo de execução(s) \\
  \hline
  \endhead
  h3016a.txt & 4952  & 30 \\
  h3016b.txt & 3375  & 30 \\
  h3016c.txt & 3750  & 30 \\
  h3018a.txt & 5648  & 30 \\
  h3018b.txt & 6303  & 30 \\
  h3018c.txt & 4445  & 30 \\
  h3020a.txt & 6538  & 30 \\
  h3020b.txt & 4998  & 30 \\
  h3020c.txt & 5811  & 30 \\
  h4016a.txt & 4640  & 30 \\
  h4016b.txt & 4946  & 30 \\
  h4016c.txt & 3805  & 30 \\
  h4018a.txt & 6046  & 30 \\
  h4018b.txt & 5436  & 30 \\
  h4018c.txt & 5439  & 30 \\
  h4020a.txt & 8452  & 30 \\
  h4020b.txt & 6287  & 30 \\
  h4020c.txt & 5846  & 30 \\
  h5016a.txt & 7533  & 30 \\
  h5016b.txt & 6092  & 30 \\
  h5016c.txt & 11435 & 30 \\
  h5018a.txt & 11221 & 30 \\
  h5018b.txt & 6743  & 30 \\
  h5018c.txt & 7810  & 30 \\
  h5020a.txt & 10657 & 30 \\
  h5020b.txt & 9971  & 30 \\
  h5020c.txt & 9070  & 30 \\
  \caption{Resultados para a Metaheurística}
  \label{meta}
\end{longtable}

\subsection{PLI}
Os resultados do PLI podem ser vistos na Tabela \ref{pli}.

\begin{longtable}{c c c c c}
  Dataset & Função Objetivo & Lim. dual & Nós explorados & Tempo de execução(s) \\
  \hline
  \endhead
  e0806.dat & 4   & 4  & 75     & 0.02   \\
  e1006.dat & 12  & 12 & 1645   & 0.79   \\
  e1008.dat & 24  & 24 & 2999   & 1.72   \\
  e1010.dat & 51  & 51 & 21849  & 21.26  \\
  e1206.dat & 22  & 22 & 67251  & 63.17  \\
  e1208.dat & 80  & 0  & 105249 & 180.01 \\
  e1210.dat & 81  & 4  & 104101 & 180.01 \\
  e1211.dat & 113 & 0  & 78081  & 180.01 \\
  e1410.dat & 125 & 0  & 81965  & 180.02 \\
  e1411.dat & 177 & 0  & 69353  & 180.01 \\
  e1808.dat & 191 & 0  & 67057  & 180.02 \\
  \caption{Resultados para o PLI}
  \label{pli}
\end{longtable}

\subsection{Branch and Bound}
Os resultados do B\&B podem ser vistos na Tabela \ref{bnb}.

\begin{longtable}{c c c c c}
  Dataset & Função Objetivo & Lim. dual & Nós explorados & Tempo de execução(s) \\
  \hline
  \endhead
  e0806.txt & 4   & 4   & 75       & 0.10 \\
  e1006.txt & 12  & 12  & 1334     & 0.10 \\
  e1008.txt & 24  & 24  & 388      & 0.10 \\
  e1010.txt & 51  & 51  & 275      & 0.10 \\
  e1206.txt & 22  & 22  & 16672    & 0.18 \\
  e1208.txt & 69  & 69  & 69305    & 0.46 \\
  e1210.txt & 81  & 81  & 116069   & 0.76 \\
  e1211.txt & 104 & 104 & 15675    & 0.21 \\
  e1410.txt & 115 & 115 & 504148   & 3.62 \\
  e1411.txt & 132 & 132 & 256840   & 2.38 \\
  e1808.txt & 135 & 135 & 12127193 & 139.70 \\
  \caption{Resultados para o Branch and Bound}
  \label{bnb}
\end{longtable}

\section{Conclusão}
A metaheurística escolhida (algoritmo genético), quando acompanhada geração a geração, mostrou
um desempenho satisfatório, conseguindo evoluir rapidamente durante as primeiras gerações. Entretanto,
em vários casos, o algoritmo fica preso em locais mínimos já após o primeiro segundo de execução,
com apenas ocasionais melhoras. Possivelmente a alteração dos hiperparâmetros pode gerar resultados melhores.

O modelo de PLI formulado se mostrou correto, atingindo resultados corretos para os casos em que
o tempo de execução foi suficiente. Entretanto, em 5 dos 11 casos, a solução ótima não foi encontrada
no tempo limite, e em 1 dos casos, a solução ótima foi encontrada, mas o limitante dual ainda não era
igual a solução para que a execução fosse encerrada.

A formulação do Branch and Bound conseguiu encontrar soluções ótimas para todos os casos dentro
do tempo alocado, indicando que nossa formulação é satisfatória. A utilização da metaheurística
também gerou bons resultados, conseguindo muitas vezes soluções relativamente boas em $100ms$.
A heurística para obtenção do empacotamento Q também conseguiu gerar bons limitantes, pois
comparando com os resultados do PLI (nos casos em que uma solução ótima foi encontrada),
o número de nós explorados é relativamente menor.

\end{document}
